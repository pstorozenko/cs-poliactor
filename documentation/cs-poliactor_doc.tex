\documentclass[a4paper]{mwart}

\usepackage[utf8]{inputenc}
\usepackage{polski}
\usepackage{anttor}
\usepackage{hyperref}


\author{Dawid Stelmach \& Piotr Pasza Storożenko}
\title{Warsztaty badawcze -- dokumentacja wstępna}

\begin{document}

\maketitle

\section{Cel projektu}

Zaprojektowanie i implementacja aplikacji wykrywającej twarz w obrazie z kamery oraz odnajdującej w zdefiniowanej bazie danych najbardziej podobną twarz. Aplikacja będzie wykorzystywała metody uczenia maszynowego i zostanie zaimplementowana w języku Python.

\section{Wymagania}

Do poprawnego działania programu konieczne będą:

\begin{itemize}
	\item Docker -- wymagania dostępne na stronie producenta, zależne od systemu operacyjnego;
	\item 1GB przestrzeni dyskowej;
	\item przeglądarka internetowa z obsługą JavaScript.
\end{itemize}

\section{Wymagania funkcjonalne}
\begin{itemize}
	\item \textbf{Podgląd obrazu z kamery} -- obraz z kamery będzie wyświetlany w czasie rzeczywistym w głównym oknie przeglądarki po lewej stronie okna w oknie o rozmiarach $500 \times 375$.
	\item \textbf{Wykryta twarz} -- wykryta twarz będzie się pojawiała w czasie zbliżonym do rzeczywistego po lewej stronie okna przeglądarki w oknie o stałym rozmiarze.
	\item \textbf{Najbardziej podobna twarz} -- twarz polityka lub aktora będzie wyświetlana w oknie o stałym rozmiarze obok twarzy użytkownika.
	\item \textbf{Bounding box} -- bounding box nie jest funkcjonalnością kluczową a co za tym idzie konieczną. Będzie dodany o ile uda się uzyskać wystarczająco niskie opóźnienie odpowiedzi z serwera.
	\item \textbf{Ilość zdjęć znanych osób}:
		\begin{itemize}
			\item aktorzy -- 2086 zdjęć, 150 mężczyzn, 150 kobiet;
			\item politycy -- 472 zdjęcia, 472 osób.
		\end{itemize}
	\item \textbf{Obsługa wielu twarzy} -- w podstawowej wersji aplikacji porównywana będzie tylko twarz która na nagraniu jest większa w rozumieniu pola liczonego w pikselach. W wersji rozszerzonej dodana będzie obsługa wielu twarzy.
\end{itemize}

\section{Wymagania niefunkcjonalne}

\begin{itemize}
	\item \textbf{Czas reakcji} -- przewidywany czas reakcji to ok $0{,}5$s w przypadku pesymistycznym, obraz wykryty odświeżany z częstotliwością ok 4x na sekundę
\end{itemize}


% \section{Źródła -- dla nas do wybrania podczas tworzenia projektu}
% \subsection{Face detection}
% \begin{itemize}
% 	\item \href{http://vis-www.cs.umass.edu/fddb/}{FDDB} -- Face Detection Data Set and Benchmark Home
% 	\item \href{https://medium.com/wassa/modern-face-detection-based-on-deep-learning-using-python-and-mxnet-5e6377f22674}{Medium: Modern Face Detection} -- z wykorzystaniem pythona i mxnet, architektura MTCNN (implementacje w innych pakietach też pewnie będą bez problemu dostępne
% 	\item \href{https://www.pyimagesearch.com/2018/06/18/face-recognition-with-opencv-python-and-deep-learning/}{pyimagesearch: face recognition} -- długi artykuł o realizacji projektu, jest też o wykrywaniu z video
% 	\item \href{https://www.hackster.io/mjrobot/real-time-face-recognition-an-end-to-end-project-a10826}{Real time face recognition} = zrobione na raspberry więc na PC pewnie nie będzie działało tak z bańki
% 	\item \href{https://realpython.com/face-recognition-with-python/}{Real python Face detection} -- pierwszy z dwóch artykułów, raczej bez szału ale widać, że proste rozwiązanie, można na początek wziąć jako prototyp
% 	\item \href{https://realpython.com/face-detection-in-python-using-a-webcam/}{Real python webcam} -- druga część, tutaj już z kamerką
% 	\item \href{https://www.pyimagesearch.com/2015/12/21/increasing-webcam-fps-with-python-and-opencv/}{Increasing FPS} -- jakiś artykuł o zwiększaniu wydajności.
% 	\item \href{https://pjreddie.com/darknet/yolo/}{YOLO} -- dobre gówno tylko nie wiem czy będzie działało dla twarzy, trzeba zrobić research
% 	\item \href{https://www.quora.com/Which-library-is-best-for-face-recognition-from-a-real-time-video-language-Python}{Quora} -- linki do kilku fajnych rzeczy
% 	\item \href{https://github.com/ageitgey/face_recognition}{face recognition github} -- pakiecik na githubie
% 	\item \href{https://github.com/cmusatyalab/openface}{open face} -- inny pakiecik
% 	\item \href{https://medium.freecodecamp.org/making-your-own-face-recognition-system-29a8e728107c}{freecodecamp} -- face recognition system
% 	\item \href{https://towardsdatascience.com/one-shot-learning-face-recognition-using-siamese-neural-network-a13dcf739e}{ine shot learning}
% \end{itemize}
% \subsection{Python wykrywanie kamery}
% w kilku z powyższych jest już ten problem uwzględniony
% \begin{itemize}
% 	\item \href{http://www.chioka.in/python-live-video-streaming-example/}{Ex1}
% 	\item \href{https://github.com/yushuhuang/webcam}{ex2}
% 	\item \href{https://benhowell.github.io/guide/2015/03/09/opencv-and-web-cam-streaming}{ex3}
% 	\item \href{https://stackoverflow.com/questions/14140495/how-to-capture-a-video-and-audio-in-python-from-a-camera-or-webcam}{ex4}
% \end{itemize}

% \subsection{Podobieństwo twarzy}
% \begin{itemize}
% 	\item \href{https://www.linkedin.com/pulse/using-deep-learning-fashion-similarity-face-vaibhav-gusain/}{LinkedIn artykuł}
% 	\item \href{http://fizzylogic.nl/2017/08/04/which-celebrity-are-you-a-celebrity-scanner-built-with-clarifai/}{celebscan}
% 	\item \href{https://blog.biolab.si/2016/11/25/celebrity-lookalike-or-how-to-make-students-love-machine-learning/}{celebrity lookalike}
% \end{itemize}

\end{document}

